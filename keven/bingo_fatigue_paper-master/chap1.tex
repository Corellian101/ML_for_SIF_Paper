%%% -*-LaTeX-*- 

\chapter{Introduction}\label{introduction}

Surface cracks are some of the most common flaws found in welds. These cracks can propagate and render the weld structurally unsound. Stress intensity factors (SIFs) are the driving force of crack growth and are essential in predicting the rate at which a crack grows through a material. Because SIFs are essential for predicting the components' fatigue life, accurate and reliable models are useful to those designing components. Exact solutions exist for calculating SIFs, but only for specific crack types, such as Irwin’s solution to an embedded elliptical crack in an infinite body \cite{tada2000stress}. For the cases that do not have exact solutions, numerical solutions are used in other crack configurations and geometries \cite{wilson1971numerical,lin1999finite,maligno2010three}. In 1971 W.K. Wilson used a boundary collocation method to calculate SIFs. Later, Lin et al. and Maligno et al. used finite element method (FEM) to numerically simulate fatigue crack growth (FCG) after calculating SIFs. Numerical methods are accurate, high fidelity predictors of SIFs but can be computationally expensive. For that reason, surrogate models are often used as low fidelity stand-ins that are much easier to compute.

Raju and Newman found numerical results through FEM for a semi-elliptical surface crack on a surface of a plate loaded in tension \cite{raju1979stress}. The results from this study were then used to create a surrogate model equation. The equations were obtained from systemically curve-fitting the SIFs to polynomials as a function of the crack's geometry \cite{newman1983stress}. Likewise, there are many developed surrogate models that map the stresses of a lower-dimensional model to 3D crack growth \cite{hombal2013surrogate,leser2016probabilistic,leser2017probabilistic}. Hombal et al. use multi-axial load histories in uncracked models to predict the crack growth of complex crack geometry. They use principal component analysis coupled with Gaussian process regression to create this surrogate model. While their method was accurate, it loses interpretability in both parts of the process. Similarly, Leser et al. train surrogate models using Gaussian process regression. We seek an interpretable surrogate model through Symbolic Regression. 

First, data is collected through high fidelity crack growth models. Fatigue cracks can be modeled using 3D continuum elements in a FE mesh using FRANC3D \cite{carter2000automated}. These cracked models are useful in calculating SIF histories that subsequently can be used to integrate crack-growth rate relations and predict the fatigue life of a part \cite{sankararaman2009uncertainty}. 

While SIFs can be calculated using 3D continuum elements, it is often computationally cheaper to use shell elements when modeling thin structures \cite{bathe2006finite}. Shell element models can include 1D elements called a line-weld representing welds connecting the edge of a surface to the face of another surface \cite{osti_1433781}. In this work, fatigue crack growth is predicted; however, representing a crack in these 1D elements is not feasible.  Consequently, we seek a methodology whereby 1D element forces can be used, along with SIFs computed from 3D continuum models, to determine equivalent SIFs for these element topologies. 

Symbolic Regression (SR) is a type of machine learning (ML) analysis used to develop surrogate models from raw data. SR is a subset of genetic programming (GP) inspired by the evolution of genes in nature \cite{banzhaf1998genetic}. SR represents symbolic equations as tree structures and iteratively mutates and crosses over the equations that perform the best on the training data. This regression process differs from classical regression in that the user does not need to input a predetermined functional form of an equation. SR has been used to find free-form natural laws using experimental data, as Schmidt and Lipson were able to show in their work using SR to implicitly find the Hamiltonian description of a double pendulum using only raw, experimental data \cite{schmidt2009distilling}. Wang et al. used raw experimental data of temperature and an order parameter as inputs to a model for system energy \cite{wang2019symbolic}. They found an explicit model for system energy as a function of temperature and the order parameter using SR. Wang et al. sought to find an established theory called Landau free energy expansion. The model found using SR looked very similar to the energy expansion; however, it was missing parts of the equation to which the output was not as sensitive. The similarities of the outputted equation and the expected equation show the importance of domain knowledge while using SR.

To validate the process of using SR as a tool in the creation of SIF equations, we present the proposed methodology on a classical problem. Raju and Newman calculated SIFs along the crack front of a surface crack in a plate loaded in tension \cite{raju1979stress}. From the collected data, they performed a systematic curve fitting by using double series polynomials \cite{newman1983stress} to fit the SIF data to an equation. First, we propose using their very same SIF data in SR to arrive at an equation that fits the data comparatively or better than their original equation. Second, we use newer and more accurate techniques in FE modeling to generate new SIF calculations. We perform SR on the newly generated data to arrive at a new equation.

In the case where SR is used to model cracks for which there is no established law, it has the advantage that a user with domain knowledge can look at the equations found and identify within an equation recognizable patterns, perhaps already known to be laws for similar cracks. This interpretability of SR is its most significant advantage \cite{otte2013safe,McConaghy2011,vladislavleva2008order}. Interpretability is vital in safety-related applications when required to ensure that the model will work outside of the data used to train the model. Often, models used for highly nonlinear data such as those developed using neural networks are mainly black boxes \cite{fujii1996bayesian}. 

Bingo \cite{bingosymbolicregression} is open-source software used for SR. The training data is passed to Bingo, and the evolution of equations is set until the equations reach a predetermined fitness or error. The resulting model is an interpretable, closed-form, mathematical model that can be used to map the line-weld forces to SIF history and, thus, predict fatigue life.
